\section{Supported Platforms}
This version of the kernel was written with support for multiple platforms in mind from the beginning. Getting a hello world running on multiple platforms was one of the earliest goals of this rewrite. Currently only two platforms are supported (x86\_64 and riscv64), but it would be nice to add more in the future, namely aarch64 and perhaps something more exotic.

The current platforms and their status are listed below.

\begin{tabular}{|r|l|p{0.3\linewidth}|p{0.4\linewidth}|}
    \hline
    \textbf{Platform} & \textbf{Qemu} & \textbf{Hardware} & \textbf{Notes} \\
    \hline
    x86\_64 & Yes & 3 machines tested, 2 work. & \\
    riscv64 & Yes & No & Support for a visionfive-2 port is planned once some bugs are ironed out in the virtio machine.\\
    \hline
\end{tabular}

\subsection{Platform Abstraction Layer}
The kernel maintains a strict boundary between platform-specific code and platform-independent code. The platform specific code is contained within the \verb|kernel/arch/xyz| and \verb|kernel/include/arch/xyz| directories, where \verb|xyz| is the target architecture. Each platform is given it's own directory, and the layout inside this directory can be considered freeform.

Each platform \textbf{must} implement the functions defined in each header file under \verb|kernel/include/arch|. These headers contain functions required by the rest of the kernel. Code within an arch directory is allowed to include headers from within it's own \verb|include/arch/xyz/| directory, however all other kernel code is only permitted to use the generic headers found in \verb|include/arch/|.

\subsubsection{Platform.h}
This file is a special case, as the platform agnostic \verb|kernel/include/arch/Platform.h| provides some complete definitions (like the core-local data struct), some utility functions and also some function declarations. These declarations serve as a guide for what a platform would need to implement to support a port of the kernel. 

Most of the declared functions can be implemented by simple inline assembly, and often these functions are made \verb|inline| (often with \verb|[[gnu::always_inline]]| too). There are some additional constants that must be defined within an implementation of this header:

\begin{itemize}
    \item \verb|size_t PageSize|: The native page size of the target platform, in bytes. Most platforms will have this as 4K (0x1000) bytes.
    \item \verb|size_t IntVectorAllocBase|: Used by the interrupt vector allocator, this sets the lowest vector number available for general use. Vectors will never be allocated below this.
    \item \verb|size_t IntVectorAllocLimit|: Largest interrupt vector allowed to be allocated (inclusive). Set this and \verb|IntVectorAllocBase| to 0 to disable the vector allocator and the ability to install interrupt handlers.
\end{itemize}

Of course platforms are free to define their own internal constants here too, for example x86\_64 defines a number of MSR and port values. Platform implementations are discouraged from including any additional headers here however, as that can lead to leaking platform specific details to other parts of the kernel.

\subsection{Platform Requirements}
The northport kernel is inteded to run on mobile or desktop class processors, and currently there is no intent to support embedded class processors.

The hardware requirements for adding a new platform are described below.

\begin{itemize}
    \item An MMU for managing the virtual address space, with the ability to distinguish between supervisor and user levels, and allow/disallow reads and instruction fetches from regions of memory. The kernel was designed with paging in mind, but anything capable of providing the functions required in \verb|Hat.h| can be supported.
    \item Interrupt control: a way of managing both internal and external interrupts is required. This includes a global mask setting, and the ability to send IPIs (in multi-core systems). Support for MSIs is nice to have, but not required.
    \item A single timer capable of sending an interrupt on a terminal count (one-shot operation), and a timer that can be used for polling. There are no strict timing requirements, but the kernel internally represents time as nanoseconds, and it can make use of more precise timers if they are available. The kernel doesn't care if these two functions are provided by the same piece of hardware or not. Only one of each type of timer is needed, and they both need to be accessible from the same processor. The system must provide a way to calibrate each timer.
    \item The kernel expects to be booted in accordance with the Limine Boot Protocol. If a compatible bootloader exists for the target platform that's easy, but if not the boot shim included in the kernel may need to be modified.
\end{itemize}

The kernel also has a soft requirement of being running on a 64-bit processor. Although it should be possible to port to a 32-bit system, this isn't a goal of the project and is yet to be tested.

While the kernel is designed to run on multi-core processors, this is not a requirement and it is perfectly capable of running on single-core systems.

\subsection{Implementation Notes}

\subsubsection{Riscv64}
The riscv ISA does not have a \verb|pause| instruction by default. It was originally added as a custom instruction by si-five and later ratified into an ISA extension. A problem arises though with the way riscv is structed: if this instruction is not specicified to your toolchain you won't be able to use the instruction mnemonic. Fortunately any good assembler can emit raw bytes into the instruction stream, so we are able to use \verb|pause|. If you see \verb|.int 0x0100000F| in any riscv code, that's the encoding for \verb|pause| (which is actually an alias for \verb|fence r, 0|). Why not use the aliased form? Well that form is actually illegal under the base spec as you cannot encode null (as in neither \verb|r| nor \verb|w|) in either the pre or post fields. An amusing oversight, which is why it's been done this way.
