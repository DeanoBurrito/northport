\section{Clocks and Timers}
\label{Clock}
Northport refers to hardware devices that provide timekeeping facilities as timers, and provides a software abstraction on top called a clock. There is only one clock in the system, and the rest of the kernel uses the clock for getting timer callbacks. Internally the clock uses whatever timers are provided by the platform to ensure it can keep the deadlines set by software.

It's important to note that the timing subsystem is 'best-effort' and \textbf{not real time}. Clock events are guarenteed to expire \textit{exactly once}, and \textit{reasonably soon} after their ideal expiry time. Reasonably soon is relative to the accuracy and precision of the hardware timers being used to drive the clock.

\subsection{Hardware Timers}
Northport has very simple timing requirements, this was done with the intent of making the kernel easier to port to other platforms. There are two types of timers we require, although only one of each is necessary.

\paragraph{Polling Timers}
A polling timer only needs to provide a software readable counter. The kernel assumes values returned from the counter increment as time passes, regardless of what the underlying timer does. If the timer counts down, you can return the inverted value (\verb|maxCount - currentCount|). The x86 PIT is one such example, which counts down: and when used for the system polling timer the returned value is inverted so the system sees an incrementing value.

\paragraph{Interrupt Timers}
Interrupt timers require the ability to set a deadline in the future, and generate an interrupt whenthe deadline is passed. Deadlines are often within the 1-1000ms range, but the kernel can take advantage of hardware that supports longer expiry times. Interrupt timers may also be referred to as \textit{sys timers} in the source code (the original term used).

The kernel doesn't interact with the various timers directly, instead it uses the functions defined in \verb|arch/Timers.h| to interface with platform specific code that manages the underlying hardware. Anything beyond the functions in that header are considered implementation details of the platform. This includes calibrating any timers that might require it.

\subsection{Software Clock}
The clock is the main timing interface for the kernel. It works by keeping a list of pending \textit{clock events}. This list is sorted with the soonest event at the head of the list, and each event tracks it's time relative to the previous event. The interrupt timer is set to the first event in the list, and the event's callback function runs is called in response to the timer expiring. The time taken for each callback to run is tracked, and any pending events that would have expired in that time are also run.

It's important to note that the callback handler runs inside the interrupt handler. Recommended practice is to queue a DPC to perform as much of the work as possible rather than running inside the interrupt handler.

In a multicore system the clock is managed by whichever core manages the hardware timers. This is usually the first core to boot (the BSP). The clock ensures that callbacks always run on the core they were queued from, regardless of which core manages the clock. For cores other than the BSP the callback is executed via the IPI mailbox mechanism.

\paragraph{Infinite Expiry Times}
If a clock event is set to expire at a time too far away for the hardware clock to encode, \textit{virtual clock events} are inserted at the clock's longest expiry time until the original event expiry time is reachable. This trades total clock duration for a little bit of memory, but allows for near-infinite expiry times (memory permitting).

\subsection{Platform Specific Details}

\subsubsection{X86}
X86 has a large number of timers available. We try the most useful timers first, and fall back to known-good timers if those fail or are not available. The timers are sourced in the following orders:

\textbf{Polling timer:} TSC, HPET main counter, PIT counter.\\
\textbf{Interrupt Timer:} Local APIC with TSC deadline, HPET comparator 0, PIT.

\subsubsection{Risc-V}
The riscv timer infrastructure is currently under-developed. We use the SBI firmware timer for interrupts, and the \verb|rdtime| instruction for polling time. Support for the ACLINT timer is planned, as is the \verb|sstc| extension.

\subsection{Example Code}
Timer functions are intended to be provided by arch-specific code for the clock's internal use, and therefore won't be documented here. Their declarations are available in \verb|arch/Timers.h| if specifics are required.

For the following few examples we'll assume we have a callback function \verb|void CallbackFunc(void*)| and some data we want to pass to it \verb|void* callbackData|. The data pointer can optionally be \verb|nullptr| if this feature is not needed. All the example functions used are declared in \verb|tasking/Clock.h|.

The following is an example of using the software clock to have a function run 10 milliseconds from now. Clock events use nanosecond precision.
\begin{lstlisting}[stringstyle=\color{black}\ttfamily]
QueueClockEvent(10'000'000, callbackData, callback);
\end{lstlisting}

To add a callback that runs every 20ms you could use the example below. Note that there's currently not a way to remove a periodic clock event, if this is something you may need it's suggested you re-queue the event yourself everytime it expires.
\begin{lstlisting}[stringstyle=\color{black}\ttfamily]
QueueClockEvent(20'000'000, callbackData, callback, true);
\end{lstlisting}

The clock also provides a function to get the current uptime of the system, measured in milliseconds.
\begin{lstlisting}
size_t uptimeMillis = GetUptime();
Log("Uptime: %lu ms", LogLevel::Debug, uptimeMillis);
\end{lstlisting}
