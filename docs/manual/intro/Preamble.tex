\section{Preamble}
First of all, thanks for taking an interest in the project! If you're somehow reading this without a copy of the source code, there are copies hosted at the following sites:

\begin{itemize}
    \item Codeberg: \url{https://codeberg.org/r4/northport}
    \item GitHub: \url{https://github.com/deanoburrito/northport}
\end{itemize}

If you encounter any bugs or issues, feel free to open an issue on either of the repository mirrors listed above. If you just want to chat about the project the best way is on discord under the handle \verb|__r4| (note the leading double underscore).

\section{Contributors}
While git tracks contributors to the project, there has been a few rewrites over time, so for posterity a complete list of all contributors is maintained here. 

\begin{itemize}
    \item Dean T, \url{https://github.com/deanoburrito}
    \item Ivan G, \url{https://github.com/dreamos82}
\end{itemize}

\section{References \& Thanks}
I wanted to say a personal thanks to the following projects and resources, and the authors behind them, as they've helped me at various points throughout the development of northport. Any code from these projects has been licensed as appropriate, while others have been useful as a point of comparison, or helpful in understanding new concepts.

\begin{itemize}
    \item The Limine Bootloader: \url{https://github.com/limine-bootloader/limine}
    \item Nanoprintf: \url{https://github.com/charlesnicholson/nanoprintf}
    \item Qoi (Quite Ok Image) Format: \url{https://github.com/phoboslab/qoi}
    \item Frigg Utils Library: \url{https://github.com/managarm/frigg}
    \item Luna Hypervisor: \url{https://github.com/thomtl/Luna}
    \item SCAL-UX: \url{https://github.com/NetaScale/SCAL-UX}
    \item Ironclad: \url{https://www.nongnu.org/ironclad/}
    \item MentOS: \url{https://github.com/mentos-team/MentOS}
    \item Mlibc: \url{https://github.com/managarm/mlibc}
\end{itemize}

The builtin terminal font is \href{https://github.com/viler-int10h/vga-text-mode-fonts/blob/master/FONTS/NON-PC/APRIXENC.F14}{APRIXENC.F14} from \href{https://github.com/viler-int10h/vga-text-mode-fonts}{this collection} of x86 text mode fonts.

\section{License}

The source for this manual and all compiled copies fall under the same MIT license as the rest of the project. For completeness, a copy is embedded below:

\begin{verbatim}
MIT License

Copyright (c) Dean T.

Permission is hereby granted, free of charge, to any person obtaining a copy
of this software and associated documentation files (the "Software"), to deal
in the Software without restriction, including without limitation the rights
to use, copy, modify, merge, publish, distribute, sublicense, and/or sell
copies of the Software, and to permit persons to whom the Software is
furnished to do so, subject to the following conditions:

The above copyright notice and this permission notice shall be included in all
copies or substantial portions of the Software.

THE SOFTWARE IS PROVIDED "AS IS", WITHOUT WARRANTY OF ANY KIND, EXPRESS OR
IMPLIED, INCLUDING BUT NOT LIMITED TO THE WARRANTIES OF MERCHANTABILITY,
FITNESS FOR A PARTICULAR PURPOSE AND NONINFRINGEMENT. IN NO EVENT SHALL THE
AUTHORS OR COPYRIGHT HOLDERS BE LIABLE FOR ANY CLAIM, DAMAGES OR OTHER
LIABILITY, WHETHER IN AN ACTION OF CONTRACT, TORT OR OTHERWISE, ARISING FROM,
OUT OF OR IN CONNECTION WITH THE SOFTWARE OR THE USE OR OTHER DEALINGS IN THE
SOFTWARE.
\end{verbatim}

\section{Target Audience}
If you're reading this, you're probably the target audience. Most of the example code shown is written in C++ (like the rest of the project), so a basic understanding of the language is required. You should also be familiar with programming in a freestanding environment.

The purpose of this manual is to document the design and implementation of the kernel and it's various interfaces (for userspace and drivers), but not external (already documented) protocols or hardware. This manual is written assuming you already understand these, or are willing to look up those materials yourself.

Having said that, if you find anything that could use clarification or otherwise lacking way please feel free to open a pull request, or simply let me know! The feedback is appreciated.

\section{Roadmap}
An up-to-date roadmap is kept in the source directory, at \verb|docs/roadmap.md|. This is where overall progress is tracked, broken down into individual features. Plans for future features are here too, but these will likely change over time.

A quick summary of the current features is also available in the project's readme, available in the root directory.

\section{Terminology}
Where possible, standard terminology is used to keep things accessible, but there are some less-standard terms and concepts used. For completeness they're described here:

\begin{itemize}
    \item \textbf{HHDM}: Higher Half Direct Map, a term borrowed from the Limine bootloader, this refers to an identity map of all physical memory that has been offset into the higher half. An identity map allows you to access physical memory at the same virtual address, and the HHDM works in a similar way but a fixed offset is added to the virtual address. This allows a full view of physical memory to be accessible without impacting the lower half. The two variables used to access the HHDM (\verb|hhdmBase| and \verb|hhdmLength|) are determined once when the kernel is booted and are constant through the kernel's lifecycle.
    \item \textbf{SUMAC}: A portmanteau of SUM (a riscv term) and SMAP (an x86 term), retconned to mean Supervisor/User Memory Access Control. SUMAC is enabled by default on systems where it's supported, and prevents the kernel from unintentionally accessing user memory. This feature is temporarily disabled when copying data in/out of the kernel address space during system calls, and otherwise is on at all times.
    \item \textbf{Extended Registers}: This term refers to any non-integer registers present in a processor. The kernel will always preserve integer registers on entry and exit, but extended registers are preserved lazily. Typically this is any floating point and vector registers that are present.
\end{itemize}
