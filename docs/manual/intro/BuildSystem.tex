\section{Build System}
\label{Build System}
The entire project is built using GNU make, core utils and some platform-specific deployment tools. 

\subsection{Requirements}
\begin{itemize}
    \item GNU Make.
    \item Core utils (rm, mkdir, cp and friends).
    \item A cross compiler: GCC and clang are both supported.
    \item Xorriso.
    \item \href{https://github.com/limine-bootloader/limine}{The limine bootloader}. A binary release is fine, no need to build from source. See their instructions for how to download and install the latest release.
    \item A system root for the target platform. GCC cross compilers include this, clang does not and you will need to source your own.
\end{itemize}

The project is built as a series of smaller sub-projects, each living in their own directory. The root makefile combines a user config file (\verb|Config.mk|) and config file specific to the target platform (\verb|misc/cross/xyz/CrossCronfig.mk|) and populates some variables for compilation. These variables include compiler and linker flags, the toolchain binaries themselves, and how the final file should be presented (as an iso, or a plain kernel elf, depending on the target's requirements).

When a subproject is built, the root makefile recursively calls \verb|make| inside the project directly, and exports the previously created variables. Each subproject then builds itself, making use of the provided compiler flags and co.

The root makefile can be thought of as a glorified config mechanism, provoding settings to each subproject. The subproject's makefile then provides the scaffold of how to actually build the subproject, and the scaffold is filled in by the variables from the root makefile.

There are some additional files \verb|misc/BuildPrep.mk| and \verb|misc/RunDebug.mk| which provide useful functionality for interacting with the project, but are not directly related to building. \verb|RunDebug.mk| provides make targets for launching and debugging the kernel inside of qemu, and \verb|BuildPrep.mk| converts the user-facing options into compiler flags and pre-processor definitions.

\paragraph{Target Triplets}
The project currently uses the reduced versions of target triplets in the form of \verb|elf-$(ISA)|, where \verb|$(ISA)| is the target instruction set. As an example, this means any riscv64 platforms would use the target triplet \verb|elf-riscv64|, regardless of their ISA string (riscv64gcv for example). For x86\_64 the target triplet is \verb|elf-x86_64|.

In practice, target triplets are used to select toolchain binaries used for GCC, and given as the \verb|--target=| parameter for clang.

\paragraph{Libraries}
So far only a single library exists, but more are planned. Common make rules and targets are defined in a separate file (\verb|misc/LibCommon.mk|) which can then be included into a library's makefile. This is to save duplicating common build code across all libraries.

\paragraph{Kernel Syslib}
The \verb|np-syslib| library is actually built twice: once with the usual flags, and a second time with kernel-appropriate flags. The kernel version (\verb|knp-syslib|) is linked with the kernel, and has more restrictions placed on the code generation (for example, it only uses integer registers). Other programs and libraries are free to use the kernel version of np-syslib, and while there is no harm in doing so it is not recommended due to the extra restrictions placed on kernel code.

\subsection{Configuration}
Configuration is done in two places: at a global level in \verb|Config.mk|, and at a platform-specific level in \verb|misc/xyz/CrossConfig.mk|, where xyz is the target platform.

The global config is where you can set the cross-compiler paths, select the compiler you want to use and other common options like your target platform. The available options are documented in the config file itself.

Each supported platform has it's own \verb|CrossConfig.mk| file, where some platform specific compiler and linker flags can be set (like the codemodel). Other actions can be taken here as well, like changing the default build target. The available options are described below.

\begin{itemize}
    \item \verb|KERNEL_CXX_FLAGS|: This variable contains all the C++ compiler flags used for the kernel. Only platform-specific flags should be added here, for flags that affect all platforms add them in \verb|Config.mk|.
    \item \verb|KERNEL_LD_FLAGS|: For adding to the kernel linker flags, similar to the above option.
    \item \verb|ARCH_DEFAULT_TARGET|: Sets the makefile target to be used for \verb|make all|. Different platforms may require the kernel/system to be in different formats, UEFI systems accept a bootable iso for example. Some other systems may only support the bare kernel binary. Currently two builtin targets are provided: \verb|binaries| just compiles the kernel binary, and \verb|iso| which creates a UEFI iso with limine and the kernel.
    \item \verb|QEMU_BASE|: Base flags and the qemu binary to use.
    \item There are some other \verb|QEMU_*| flags that are added to the base flags depending on which make target is used (KVM/no KVM, UEFI/no UEFI).
\end{itemize}

\subsection{Make Targets}
The following make targets are available in the root makefile (i.e. from the project root directory). This is the intended interface into the build system.

While GNUmake is used for the build system, northport \textbf{does not} make use of dependency files (\verb|.d|) currently, so changes to \verb|.cpp| source files will trigger a rebuild of that file, changes to a \verb|.h| header file will not trigger rebuilds in affected source files. To test changes made to a header file, the recommended approach is to run \verb|make clean| followed by \verb|make| to perform a clean rebuild.

\begin{itemize}
    \item \verb|make|: Prints the help text, showing a summary of available make targets and notable files.
    \item \verb|make all|: Performs an incremental build of everything, and then performs a platform-specific packaging step. This might include burning a bootloader to an iso file, or leaving the kernel binary as is.
    \item \verb|make clean|: Completely removes all build files, resulting in the next build starting from scratch.
    \item \verb|make run|: Creates a platform-specific package, and launches it in qemu for the target architecture. By default this tries to use KVM.
    \item \verb|make run-kvmless|: Same as above, but explicitly disables use of KVM.
    \item \verb|make debug|: Similar to \verb|run|, but halts the virtual machine and starts the qemu gdb server with default arguments.
    \item \verb|make attach|: The other half of \verb|debug|, launches gdb with the kernel symbols loaded, and attempts to connect the qemu gdb server (default settings: tcp/1234).
    \item \verb|make docs|: Renders the documentation as a pdf.
    \item \verb|make docs-clean|: Removes temporary build files related to docs, useful after updating latex toolchain or a buggy build.
\end{itemize}

\subsection{Development Cycle}
For the initial setup you'll need to clone the repository, install any missing tools from the list above, and then adjust the global config (\verb|Config.mk|) to suit your environment. After opening a shell in the project root directory, and executing \verb|make run| the project should build and qemu should launch with the kernel. If this works then you're good to go.

General development is straightforward, \verb|make run| can be used to build the latest changes and test them. If you need to debug you can use \verb|make debug| in one terminal to launch qemu and \verb|make attach| in another to launch and attach gdb. As mentioned previously header files are not tracked as dependencies, so a clean build may be required when editing those.

Performing a clean build is also required when changing the target architecture, as the ISA the file was compiled for is not tracked. If a clean build is not performed you may encounter linker errors.
